\section{Performance Requirements}
%TODO
myTaxiService will perform 95\% of the operations within 4 seconds; the total amount of the operations within 10 seconds.
The system should ensure at least 2000 passangers connected and 500 taxi drivers connected.

\section{Design constraints}
myTaxiService wants to reach most of taxi drivers and passengers, requiring minimum specifications for devices.
Taxi drivers, registered to the system, have to use their own devices provided with GPS navigation system to perform the service. Mobile applications have to offer backward compatibility.

\section{Software system attributes}
\paragraph{Reliability}
The mean time between failures (MTBF) shall exceed 3 months.
\paragraph{Availability}
In order to maintain the system up-to-date and secure, myTaxiServer schedules downtime periods where will be executed routine operations. The service should be available 99\% of the time.
\paragraph{Security}
myTaxiService to ensure service availability and data protection use:
\begin{itemize}
	\item AES cryptography algorithm on network operations
	\item Data are encrypted and stored in backup drives to prevent system failure
	\item Login authentication. Users, after the registration, have to confirm their e-mail with the security code sent to the e-mail write in the registration form
	\item SQL injection detection
\end{itemize}
Server architecture will be implemented separating
data from application. Application server must be separated
from database and from the web server. All architectures are divided by firewalls.
\paragraph{Maintainability}
To ensure an easy maintenance of the software, it must be well-documented and written following coding patterns.
\paragraph{Portability}
Web programming ensures a wide target of browser. Mobile applications instead, cause of different languages and devices, have to be written following coding patterns for easy portability.
Availability of the service is ensured by hardware and software limitations in Section 2.4.2.

