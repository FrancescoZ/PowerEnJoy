\documentclass[]{report}
\usepackage{graphicx}
\usepackage[dvipsnames]{xcolor}
\usepackage{listings}
\usepackage{soul}
\usepackage{titling}

\renewcommand{\listfigurename}{Figure Contents}

\title{{\Huge\textit{Power EnJoy}}\\{\LARGE Requirements Analysis and Specification Document}}
\author{Redaelli Marco 877622, Zanoli Francesco 877471}

\pretitle{
  \begin{center}  
 \includegraphics[width=6cm]{Resources/PoliLogo.png} 
 
 }
\posttitle{\end{center}}

\begin{document}
\maketitle

\tableofcontents

\listoffigures
\chapter{Introduction}

This document represent the Requirement Analysis and Specification Document (RASD). The main goal of this document is to completely describe the system in terms of functional and non-functional requirements, analyse the real need of the customer to modelling the system, show the constraints and the limit of the software and simulate the typical use cases that will occur after the development. This document is intended to all developer and programmer who have to implement the requirements, to system analyst who want to integrate other system with this one, and could be used as a contractual basis between the customer and the developer.

\section{Scope}
\PowerEnJoy{} is a car sharing service developed to fit all the reality, as small or large city.
The main goals of the system are:
\begin{itemize}
	\item Guarantee the access of driver to the service
	\item Guarantee a fair management of reservation and use of electric car
\end{itemize}
The system architecture will be a four-tier architecture: client, server application, car system and server database. It will be created by using the MVC architectural pattern.
The system will be divided into components with respect to the principles leading to good design:
\begin{itemize}
	\item Each individual component will be smaller in order to be easier to understand
	\item Coupling will be reduce where possible
	\item Reusability and flexibility will be increase in order to make easier future implementation
\end{itemize}
The system will have efficient algorithm in order to increase its performance;
in the document will be given special attention to the geolocalization algorithm.


\section{Definitions, acronyms and abbreviations}
\paragraph{Definitions}
\begin{itemize}
	\item User: Someone registered to the system
	\item Visitor: user that is not registered nor logged in
	\item System: the union of software and hardware to be developed and implemented
	\item Parking area: it is a reserved area, predefine by the system, where I can park the car but I cannot recharge it.
	\item Safe area: it is a reserved area, predefine by the system, where I can park the car and plug it into charge.
	\item Free car: The car is visible on the map and available for a reservation
	\item Reserved car: The car is not visible on the map and the user who reserved it didn't access yet.
	\item Internet: The Internet is the global system of interconnected computer networks that use the Internet protocol suite (TCP/IP) to link billions of devices worldwide.
	\item Service: The functionalities offered by myTaxiService via software appli- cations to passengers and taxi drivers.
	\item System: The whole hardware and software parts that together deliver myTaxiService as a service to passengers and taxi drivers.
	\item Smartphone: A mobile phone with an advanced mobile operating sys- tems, which combines features of a personal computer operating system with other features useful for mobile or handheld use.
	\item Application: A computer program (i.e. a piece of software) that per- forms a group of coordinated functions, tasks, or activities for the benefit of the user.
	\item World Wide Web: Also referred as web, is an information space where documents and other web resources are identified by URLs, interlinked by hypertext links, and can be accessed via the Internet.
	\item Web Page: A web document that is suitable for the World Wide Web and the web browser.
	\item Web Server: An information technology that processes requests via HTTP and stores, processes and deliver web pages to clients.

\end{itemize}
\paragraph{Acronyms}
\begin{itemize}
	\item RASD: requirements analysis and specification document
	\item AES: Advanced Encryption Standard
	\item FIFO: First In First Out
	\item ETA: estimated time of arrival
	\item API: application programming interface
	\item GPS: Global Positioning System
	 IEEE: Institute of Electrical and Electronics Engineers.
	\item SRS: Software Requirements Specification, document based on IEEE 830.
	\item RASD: Requirements Analysis and Specification Document, also known as SRS.
	\item UML: Unified Modeling Language.
	\item Wi-Fi: Wireless Fidelity, technology for local area wireless computer networking technology based on IEEE 802.11 standard.
	\item WiMAX: Worldwide Interoperability for Microwave Access, technology based on IEEE 802.16 standard.
	\item API: Application Programming Interface, a set of public accessible func- tions, protocols and tools provided by a specific application for building software applications.
	\item GPS: Global Positioning System.
	\item DBMS: Database Management System.

\end{itemize}


\section{References}
\begin{itemize}
	\item Software Engineering 2 Project AA 2016/2017: Assignments AA 2016-2017	
\end{itemize}

\section{Overview}
This document is essentially structured in four parts:
\begin{itemize}
	\item Introduction: Overview of the RASD document. Specifically provides both a description of the software, and a set of information about the organization of the document.
	\item Overall Description: Provides a high-level description of the software requirements, not describing specifically the main aspect of them, but only providing a background of those requirements. The main purpose of this section is make the requirements easy to understand.
	\item Specific Requirements: Shows all of software requirements to a level of detail sufficient to enable designers to design a system to satisfy those requirements, and testers to test the system requirements.
	\item Appendix: it provides informations that are not considered part of the actual RASD. It includes: software and tools used, alloy implementation, project group organization
\end{itemize}


\chapter{Overall description}

\section{Product perspective}
The PowerEnJoy system is composed by a mobile application and a web server. The mobile application runs on iOS, Android and Windows Phone. In plus on the web server, the system give a notion about how it works and an answer to the frequently asked question. Additional functionalities are provided through the use of APIs or interfaces.

\section{Product functions}
The system allows different kinds of user to perform different actions. In particular:
\begin{itemize}
	\item Visitors can simply register or log in.
	\item Logged user can reserve and rent a car, plug a car into a safe area and finally communicate with the customers service in case of any problems during one of this operations
\end{itemize}
The user's action as the all the sequence is it possible to do with them is explain in the figure \ref{fig:UseCasesDiagram}
\begin{figure}
   \begin{center}
    \includegraphics[width=\textwidth]{Resources/UML.png}
    \caption{Class Diagram}
   \end{center}
    \label{fig:ClassDiagram}
\end{figure}
\vfill
\vfill


\section{User characteristics}
The system wants to give to the users an easy way to interact with it. This without undermine the complexity of the problem and achieving all the goal imposted by the stakeholder. To do that, users must be able to install the mobile application from the store, besides, their phone must be provided with a GPS, a camera and an internet connection.
\section{Constraints}
\subsection{Regulatory policies}
PowerEnJoy is a service provided by the public company responsible for public transportation in the city. The user, who reaches this service by web or mobile application, has to agree to License Agreement rather than Privacy policy and Terms of use at registration.\\
The user access and use of the services constitutes his/her agreement to be bound by these Terms, which establishes a contractual relationship between him/her and PowerEnJoy. If user does not agree to these Terms, he/she may not access or use the services. PowerEnJoy may immediately terminate these Terms or any services with respect to him/her, or generally cease offering or deny access to the Services or any portion thereof, at any time for any reason.\\
PowerEnJoy collects the information provided by the user, for example when creating or making changes to services on demand, through contact with customer service or during other communications. This information may include: name, email, phone number, mailing address, profile picture, payment method, products required (for service delivery), delivery receipts and other information user choose to provide. The personal data will be used only to provide the services requested.\\
User is responsible for obtaining the data network access necessary to use the services. User mobile network's data and messaging rates and fees may apply if he/she accesses or uses the services from a wireless-enabled device. User is responsible for acquiring and updating compatible hardware or devices necessary to access and use the service and applications and any updates thereto.\\
PowerEnJoy does not guarantee that the services, or any portion thereof, will function on any particular hardware or devices. In addition, the services may be subject to malfunctions and delays inherent in the use of the Internet and electronic communications.

\subsection{Hardware limitations}
PowerEnJoy defines the minimum requirements for using web and mobile applications.
\begin{itemize}
	\item \textit{Web application}\\
	Supported minimum version browsers: Chrome 25, Internet Explorer 10, Firefox 20, Safari 25. Other browsers may also work\\
	Web access at the minimum speed of 1Mbps
	\item \textit{Mobile application}\\
	Operating system: Android, iOS, Windows Phone\\
	Memory: 512MB RAM\\
	Hard drive: 50MB of free space\\
	GPS navigation system (only for taxi drivers)\\
	Web access at the minimum speed of 1Mbps
\end{itemize}

\subsection{Interfaces to other applications}
PowerEnJoy provides APIs to enable development of additional software on this platform.

\subsection{Parallel operation}
PowerEnJoy supports parallel operations cause of the nature of service. Many users can access to the service at same time thus system and database have to work with parallel requests.

%\subsection{Audit functions}
%\subsection{Control functions}

\subsection{High-order language requirements}
PowerEnJoy requires the following high-order languages based on different platforms.
\begin{itemize}
	\item\textit{Web}\\
	HTML 5 and CSS 3 standards.
	\item\textit{Android}\\
	Java 8
	\item\textit{iOS}\\
	Swift 2.0
	\item\textit{Windows Phone}\\
	C\# 6.0
	\item \textit{Server}\\
	MySQL 5.6.19 and PHP 5.6.7
\end{itemize}

%\subsection{Signal handshake protocols}

\subsection{Reliability requirements}
PowerEnJoy relies on network connections thus reliability issues are equivalent to performance issues.  However, the application should not corrupt server data as a result of its actions. The system has to guarantee whole-time availability.

\subsection{Criticality of the application}
 PowerEnJoy relies on network systems and servers. Scheduled downtime is acceptable. This system requires a generator backup and redundant power in the event of failover.

\subsection{Safety and security considerations}
PowerEnJoy guarantees secure communications through AES encryption algorithms.
\section{Assumptions and dependencies}

\section{Apportioning of requirements}

\chapter{Specific requirements}
In this chapter are analysed all the requirements related to the system. Each section corresponds to a specific category of requirement. %In particular the section \emph{"Functional Requirement"} explain for each situation the mandatory requirements that the system has to implement to guarantee its correct workings. 

\section{External interface requirements}
In this chapter are analysed all the requirements related to the system. Each section corresponds to a specific category of requirement. In particular the section \ref{sec:ExternalInterface} explain the requirements and the constraints the system must to respect to achieve the goal imposed by the customer, as the user interface.
\section{External interface requirements}\label{sec:ExternalInterface}
\subsection{User interfaces}
The interface of PowerEnJoy can be both for web application and mobile application. From figure \ref{fig:LogPage} to figure \ref{fig:CarInfoPage} are presented some of the most important pages and screens of PowerEnjoy. In particular in figure \ref{fig:MapPage} is showed the home page of the PowerEnJoy's application, it is composed by a map and some marker located on the map depending of the position of the car. It important to understand that only the car in a free state are showed within the map. The figure \ref{fig:CarInfoPage} appear whereas when the user click on a marker in the map and shows the information of the selected car and a button to reserve it.
	\begin{figure}[H]
		\centering
		\includegraphics[height=8cm]{Resources/LoginMockup.pdf}
		\caption{Login page}
		 \label{fig:LogPage}
	\end{figure}
	\begin{figure}[H]
		\centering
		\includegraphics[height=8cm]{Resources/RegistrationMockup.png}
		\caption{Registration page}
		 \label{fig:RegistrationPage}
	\end{figure}
	\begin{figure}[H]
		\centering
		\includegraphics[height=8cm]{Resources/MapMockup.pdf}
		\caption{Home page or Map page}	
		 \label{fig:MapPage}	
	\end{figure}
	\begin{figure}[H]
		\centering
		\includegraphics[height=8cm]{Resources/CarInfo.png}
		\caption{Car information page}	
		 \label{fig:CarInfoPage}
	\end{figure}

\section{Use Cases}
In this paragraph some use cases will be described. These use cases can be derived from the scenarios and the use case diagram.

\begin{figure}
   \begin{center}
    \includegraphics[width=\textwidth]{Resources/UseCaseModel.png}
    \caption{Use cases diagram}
   \end{center}
    \label{fig:UseCasesDiagram}
\end{figure}

\begin{center}
\line(1,0){250}
\end{center}

\begin{description}
	\item[Name:] User registration
	\item[Actors:] Visitor
	\item[Entry conditions:] There are no entry conditions
	\item[Flow of events:]  \ \\
		\begin{itemize}
			\item The visitor arrive to the home page of the application, as is not logged in is redirect to the login/registration page
			\item The visitor enter his personal information, his driver license, a photo of his driver license and some payment method
			\item The visitor clicks on the confirm button
			\item The application suggest the user to read his emails to receive the password
			\item The visitor login after read the password
		\end{itemize}
	\item[Exit conditions:] The visitor is redirect to the home page of the application
	\item [Exception:] The information furnished by the visitor are not correct or ambiguous as the following case:
		\begin{itemize}
			\item The Email has not the correct format
			\item The Birthday is not at least eighteen years ago
			\item The Payment method is not valid
			\item The information's of the driver license don't correspond with the information furnished by the visitor
			\item The Driver license is not valid
		\end {itemize}
		Also the visitor could had forgot to enter some requested camp or to accept the Terms and Conditions. In all this case, the system does not send any mail to the visitor but notifies him that an error has been made and allows to input the incorrect data again
	\end{description}
	
\begin{center}
\line(1,0){250}
\end{center}

\begin{description}
	\item[Name:] User Login
	\item[Actors:] User
	\item[Entry conditions:] There are no entry conditions
	\item[Flow of events:]  \ \\
		\begin{itemize}
			\item The user arrives at the Login page of the mobile application.
			\item The user inputs his email address and his password.
			\item The user clicks on the log in button.
			\item The system redirects the user to the home page.
		\end{itemize}
	\item[Exit conditions:] The user is successfully redirected to the application home page.
	\item [Exception:] The email and/or the password furnished by the user are not correct. In this case, the system does not redirect the user to the home page but notifies him that an error has been made and allows to input his email and password again. The user can also forget his/her password, in this case he/she can ask to generate another password and received it on his/her personal email address.
\end{description}

\begin{center}
\line(1,0){250}
\end{center}

\begin{description}
	\item[Name:] Reserve a car
	\item[Actors:] User, 
	\item[Entry conditions:] There is at least a car not reserved neither used.
	\item[Flow of events:]  \ \\
		\begin{itemize}
			\item The user arrives at the home page of the application that shows the map with the markers of the cars.
			\item The user choose a car.
			\item The user clicks on the marker of the car chosen.
			\item The user clicks on the "Reserve" button
			\item The application shows to the user the time remained to start the engine and the position of both the actors.
			\item The user arrives next to the car.
			\item The user clicks on the button "Open the car"
			\item The car is opened by the system
			\item The user get into the car and start the engine.
		\end{itemize}
	\item[Exit conditions:] The user successfully start the engine of the car
	\item [Exception:] Two users reserve the same car in a really small difference of time, the system in this case will delete the reservation that is requested later.
\end{description}

\begin{center}
\line(1,0){250}
\end{center}

\begin{description}
	\item[Name:] End a rent
	\item[Actors:] User, Car 
	\item[Entry conditions:] The car is in the state In Use
	\item[Flow of events:]  \ \\
		\begin{itemize}
			\item The user stop the engine of the car and get out of the vehicle.
			\item The car check if all the doors are closed and there is no one into the it.
			\item The system close the car
			\item The system wait 5 minute
			\item The system send the request for payment to the external system
			\item The external system answer to the request
			\item The system notify the user about the end of the rent with a message on the app
		\end{itemize}
	\item[Exit conditions:] The user successfully end the rent
	\item [Exception:] 
		\begin{itemize}
			\item The car is not parked in predefined area, in this case the system will not allow the user to end rent and close the car. 
			\item The payment request failed, in this case the system block the user and he/her will not be able to use the system  again until he complete the payment
			\item The car is stopped but the user still inside or a door is open, in this case the rent will not end and the user is charged as he/she would still driving.
		\end{itemize}
\end{description}

\begin{center}
\line(1,0){250}
\end{center}

\begin{description}
	\item[Name:] Profile settings
	\item[Actors:] User 
	\item[Entry conditions:] The user needs to be logged into the application
	\item[Flow of events:]  \ \\
		\begin{itemize}
			\item The user click on profile in the menu of the application
			\item The user change his/her information that have changed
			\item The system notify the user that the settings have been successfully updated.
		\end{itemize}
	\item[Exit conditions:] The user successfully save his/her new settings
	\item [Exception:] The information furnished by the user are not correct or ambiguous as the following case:
		\begin{itemize}
			\item The Email has not the correct format
			\item The Birthday is not at least eighteen years ago
			\item The Payment method is not valid
			\item The information's of the driver license don't correspond with the information furnished by the visitor
			\item The Driver license is not valid
		\end {itemize}
\end{description}


\section{Functional requirements}
\subsection{Registration}
Visitors can register to PowEnJoy through mobile application. This operation requires the visitor to fill a registration form with personal data and accept PowEnJoy terms and conditions, including personal data policies, according to local law. The system requires the visitor personal information as name, surname, and birthday, payment information ( as a credit card or a paypal account) and proof of the possesion of a valid driver license
If any of the previous requirements are not met or any input is invalid, the registration fails and the system asks the visitor to repeat the process. Other- wise, a verification email containing the password of the account is sent to the provided email address. To validate his account the visitor needs to login one time with the provided password.
\paragraph{Scenario}
Meg is a student. She has heard about PowEnJoy and, finding it an easy and ethical way to travel, wants to subscribe to it.
Therefore, she download the mobile application from the store and clicks on Register in the main screen. She fulfil the form, accepts the term and conditions  and she click Confirm. However, the system cannot verify Meg's driver license because she forgot to put the photo that prove the possession of it. It therefore asks Meg to take the picture from her mobile's camera. Once she has enter everything correctly she click on Confirm, this time the application valid his credential and tell to meg to check her emails, she will find the confirmation of the correct registration and the given by the software. Meg read her emails and can finally open the application again and login with the given password and the email she gave before.
\paragraph{Diagrams}
\paragraph{Special Requirements}
\begin{itemize}
	\item Visitor can abort the registration process at any time.
	\item The password in the email must be used within 1 day, otherwise the registration is deleted along with the visitor?s info.
	\item Registration form contain the following information (fields):
	\begin{itemize}
		\item Email address.
		\item First name.
		\item Surname.
		\item Address.
		\item City.
		\item Postal Code.
		\item Credit card code.
		\item Expiration date of the credit card.
		\item Secure code of the credit card.
		\item Driver license code.
		\item Expiration of driver license.
		\item Photo of a driver license.
	\end{itemize}
	\item Email address cannot be the same as ones from other PowerEnJoy users.
	\item The photo of the driver license must be taken by the camera of the mobile.
\end{itemize}

\subsection{Login}
Visitors on PowerEnJoy mobile application may access to an existing registered user account providing its corresponding email address and password. In case the submitted info do not match with any existing account info, the system notifies the visitor that the email address doesn?t exist, or that it exists, but the submitted password is wrong. In case a user forgets his/her password, the system allows him/her to retrieve it, automatically creating a new password, setting it as the user?s one and sending it to the provided email address.
\paragraph{Scenario}
\begin {enumerate}
\item Freddy is user of PowerEnJoy. He already downloaded the application from the store and he has already done the registration from the application. He cannot remember the password given from the system during the registration time. Therefore he open the app on the home page and he is redirect to the login page. He click then on the forget password link and the application ask him his email address. He insert the email address and then the application show another message telling the user to check the emails. Once he has received the new password, Freddy can finally open again the application and login with the his email address and the new password.
\item Eleonor is a lawyer familiar with the PowerEnJoy system, she have recently changed phone and she has already download the application again. She open the application and she is redirect in the login page. she fills both fields and clicks on ?Log in?. The system verifies her info: the operation ends successfully, and she gains access to the user homepage.
\end{enumerate}
\paragraph{Diagrams}
\paragraph{Special Requirements}
\begin{itemize}
	\item Visitors must fill the "email field" with an existing email address in order to successfully log in.
	\item Visitors must fill the ?password? field with the only password correspond- ing to the submitted email address in order to successfully log in.
	\item The system will ignore log in requests if at least one of the ?email? and ?password? fields are left blank.
	\item The system allows visitors to retrieve their password if they forget it, by clicking ?Forgot password??.
	\item The system requires visitors to submit an existing email address in order to retrieve their password.
	\item The system will take care of assigning the user a new password, when he/she states to have lost the previous one.
	\item The system will take care of sending to the email address submitted by the visitor the new assigned password, when he/she states to have lost the previous one.
	\item The system allows visitors to retrieve their password once a day.
	\item The system remember the user's credential until the user decide to logout.
\end{itemize}

\subsection{Reserve car}
Logged user on PowerEnJoy can look for cars near his/her position, or next to a specify address, and reserve one for a rent. This operation is possible using the map on the home page of the application that indicate with a marker the position of the car, only the cars that are in a free state can be reserved and are visible on the map
\paragraph{Scenario}
Francis needs to go home from a dinner with his friends. It is late and there are public transport anymore. He is already registered and successfully logged-in in the PowerEnJoy application. He decided to reserve a car using the application. He opens the application and he is directly redirect to the application home page that contains the map with the markers of cars near him. He choose a marker and he click on it. The app show him the information of the car as its battery charge and its position. The car is really close to him therefore he click on the button reserve and he moves next to the car. Meanwhile the application shows him the  a timer, the vehicle registration plate, his position and the position of the car. Once the Francis arrives next to the car the app shows him a button to open the car. The rent start when Francis start the engine of the car.
%Another scenario with a men who can't reach the car until in one hours?
\paragraph{Diagrams}
\paragraph{Special Requirements}
\begin{itemize}
	\item A car change is state from \emph{"Reserved"} to \emph{"In use"} only when the engine starts
	\item A car can be reserved and showed on the map only if its state is \emph{"Free"}
	\item A car stays in the \emph{"Reserved"} state for at maximum one hour, if it's not picked-up it return to the state \emph{"Free"}
	\item Each car have a precise position
	\item An user can open the car through the app only if it is near to it
	\item Each user can reserve only one car at the same time
	\item A car can be reserved from only one user at the same time
\end{itemize}

\subsection{End a rent}
Once the user has finish his/her ride he/she have to parks the car in a safe area or a parking and stop the engine. The payment of the rent as its end are automatically done by the system itself. The end of a rent can also provide some discount on the final payment or the add of some overtaxes.
\paragraph{Scenario}
\begin{enumerate}
	\item Isa is an habitual user of PowerEnJoy. She picked-up a car to cross the city and be ecologic with the system. She has finished her ride and she wants to end the rent. She parks the car near a safe area near her destination and she stops the engine of the car. The safe area, as defined, has a plug to recharge the car so Isa plugs the car just after she has stopped the car. The system notify Isa of the correct end of her rent and show her the final bill, that contain a discount of 30\%, with a message on the application.
	\item Laura took a car of PowerEnJoy to get home with her family, her husband and her tow child. She park the car in a parking but unfortunately the battery of the car is at 10\% and there are no safe area next to Laura's house, the most near is 3.2 Km away. Laura end her rent stopping the engine of the car and receive a message from the app that show her an overtaxes of 30\% due to the position of the parking and the state of the battery life of the car.
	%Another scenario?
\end{enumerate}
\paragraph{Diagrams}
\paragraph{Special Requirements}
\begin{itemize}
	\item If the system detects the user took at least two other passengers onto the car, the system applies a discount of 10\% on the last ride.
	\item If a car is left with no more than 50\% of the battery empty, the system applies a discount of 20\% on the last ride.
	\item If a car is left at special parking areas where they can be recharged and the user takes care of plugging the car into the power grid, the system applies a discount of 30\% on the last ride.
	\item If a car is left at more than 3 KM from the nearest power grid station or with more than 80\% of the battery empty, the system charges 30\% more on the last ride to compensate for the cost required to re-�?charge the car on-�?site.
	\item The car has to be in the same position of a safe area or a parking
	\item The user has five minutes to plug the car if he/she wants a discount
\end{itemize}

\subsection{Report problems}
Every logged in user can report a problem to the PowerEnJoy team during the all time of use of the system. In particular the user has a button to immediately contact the customer service during:
\begin{itemize}
	\item The reservation of a car, in case the car is not opened by the system.
	\item The rent of a car, in case of accident or problem due to the system.
	\item The charge of a car, in case some safe area is not working correctly.
	\item The payment, in case of some error appeared during the payment time.
\end{itemize}
\paragraph{Scenario}
\begin{enumerate}
	\item Marc is a logged user who has already reserved a car. The rented car is really near to him and Marc wants to open it. Unfortunately the board computer of the chosen car, the one that able the system to open it, it is broken.Therefore he decides to call the customer service with the button in the reservation page. The customer service office answer to his call and let Marco abort his reservation without paying any additional feeds.
	\item Claire is a user of PowerEnJoy who is renting a car, during her ride she rear-end another vehicle. Unaware about the procedure to follow she open the PowerEnJoy application and she click on "Customer service" in the menu. An employ of PowerEnJoy system answers and explains all the document Claire needs to complete before end her rent with the procedure to follow in her case. The operator opens also a Intervention request with the third-part company who is responsible to maintain and repair the cars of the system.
	\item Jack is using a PowerEnJoy's car and he wants to park it because he is near to his destination. The battery charge is under 20\% and he wants to plug the car in charge in order to let the next user able to use it for a longer period of time. Unfortunately the nearest safe area is broken and Jack cannot plug the car into charge. To not income in overtaxes he decides to call the Customer service. The employ answers to Jack's call and report the problem to the third-part agency. In plus the PowerEnJoy employ preserve Jack to receive overtaxes on his last rent but he also block every type of discount.
\end{enumerate}
\paragraph{Diagram}
\paragraph {Special Requirements}
\begin{itemize}
	\item The user must be able to contact the customer service 24h/24h
\end{itemize}

\subsection{Profile settings}
The system allows logged in users to view and modify their profiles at any moment, as long as they?re logged in. While modified email addresses, driver license or payment method must be unique in all the system, otherwise the system denies the modification request. In case of modified email address, the system sends a confirmation email to the new address. Modification will successfully ends when the user clicks the link in the sent email.
\paragraph{Scenario}
\begin{enumerate}
	\item  Zac uses to periodically change his account password, in order to increase protection. To do so, every 3 months, he opens PoweEnJoy on his mobile phone, chooses ?Profile?, then ?Modify?. He selects the password field, writes down a new one, then writes it again in the ?Confirm password? field. Finally, he clicks ?Confirm?: the system informs him that his account password has successfully been updated.
	\item Sailor is a user of PowerEnJoy ans she has recently change her credit card because it was expired. She needs so to open PoweEnJoy on his mobile phone, chooses ?Profile?, then ?Modify?. She selects the old credit card and she writes all the new information about her new payment method. Finally she clicks "Confirm" and the system informs her that the account payment has been successfully updated.
\end{enumerate}
\paragraph{Diagrams}
\paragraph{Special Requirements}
\begin{itemize}
	\item Account settings are accessible from the start screen of both apps, through the ?Profile? button.
	\item The system allows users to view all their profile info, submitted during registration
	\item The system allows users to modify all their profile info, submitted during registration.
	\item Modifying the password requires to write the old one, and the new one twice; if the former password is not correct or if the two new passwords submitted do not match, the system asks for all passwords again and notifies the user.
	\item Modifying the email address, the driver license requires that the new one doesn?t match with the one of another registered user.
	\item Modifying the email address requires confirmation through an email sent to the submitted email address.
	\item The system allows users to abort modifications at any time.
	\item The system allows users to delete their account: confirmation is required to proceed.
\end{itemize}


\section{Performance Requirements}
%TODO
myTaxiService will perform 95\% of the operations within 4 seconds; the total amount of the operations within 10 seconds.
The system should ensure at least 2000 passangers connected and 500 taxi drivers connected.

\section{Design constraints}
myTaxiService wants to reach most of taxi drivers and passengers, requiring minimum specifications for devices.
Taxi drivers, registered to the system, have to use their own devices provided with GPS navigation system to perform the service. Mobile applications have to offer backward compatibility.

\section{Software system attributes}
\paragraph{Reliability}
The mean time between failures (MTBF) shall exceed 3 months.
\paragraph{Availability}
In order to maintain the system up-to-date and secure, myTaxiServer schedules downtime periods where will be executed routine operations. The service should be available 99\% of the time.
\paragraph{Security}
myTaxiService to ensure service availability and data protection use:
\begin{itemize}
	\item AES cryptography algorithm on network operations
	\item Data are encrypted and stored in backup drives to prevent system failure
	\item Login authentication. Users, after the registration, have to confirm their e-mail with the security code sent to the e-mail write in the registration form
	\item SQL injection detection
\end{itemize}
Server architecture will be implemented separating
data from application. Application server must be separated
from database and from the web server. All architectures are divided by firewalls.
\paragraph{Maintainability}
To ensure an easy maintenance of the software, it must be well-documented and written following coding patterns.
\paragraph{Portability}
Web programming ensures a wide target of browser. Mobile applications instead, cause of different languages and devices, have to be written following coding patterns for easy portability.
Availability of the service is ensured by hardware and software limitations in Section 2.4.2.



\appendix

\chapter*{Revision}
In the following are listed the differences between versions:
\begin{enumerate}
	\item First version
\end{enumerate}

\chapter*{Hours of work}
In the following are listed the hours of work that each member of the group did:
\begin{enumerate}
	\item Marco Redaelli:  \emph{hours}
	\item Francesco Zanoli: \emph{hours}
\end{enumerate}

\end{document}
       
