The system allows different kinds of user to perform different actions. In particular:
\begin{itemize}
	\item Visitors can simply register or log in.
	\item Logged user can reserve and rent a car, plug a car into a safe area and finally communicate with the customers service in case of any problems during one of this operations
\end{itemize}
The user's action as the all the sequence is it possible to do with them is explain in the figure \ref{fig:UseCasesDiagram}

\subsection{Goals}
Here are described the software's high level goals:
\begin{enumerate}[label=\subscript{G}{\arabic*}]
	\item Allow visitor only to sign-up or sign-in
	\item Allow user to log in
	\item Allow user logged in to rent a car
	\item Allow user logged in to book a car in a certain location
	\item Allow user  logged in to see the reservation's confirmation and the time of expiration
	\item A non registered users can only register once to the service.
	\item A registered user can login to the service only when not logged in.
	\item A registered user can logout from the service only when logged in.
	\item A user who request a rent can abort the process when ever he/she wants
	\item A user can get discount or overtaxes from his/her last rent
	\item Further services can be built on the top of the existing one through a set of given APIs.
\end{enumerate}

\subsection{UML}	
\showImage{Resources/UML.png}{UML Class Diagram}{fig:ClassDiagram}
\newpage