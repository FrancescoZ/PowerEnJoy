In this section are reported all the coding choices that do not meet the \textbf{Code Inspection Checklist} given.

\section{Naming Conventions}
\begin{itemize}
  \itemBold{Checklist[1]}
    \begin{itemize}
      \item The method
        \renderPartialCode{61}{61}
        has a name that is not really meaningful. In fact, the method
        do a search for the product and find the ones the user can be interested about.
        The random contained in the name is misleading and  it refer to the random delete that it is done
        if the algorithm find more than three elements.
      \item the variable' name
        \renderPartialCode{160}{160}
        contain a grammar error, it should be \textbf{Occurrences}
      \item The variables' name
        \renderPartialCode{187}{187}
        \renderPartialCode{195}{195}
        \renderPartialCode{208}{208}
        \renderPartialCode{268}{268}
        is not really clear and should be more specified with a comment or with a more meaningful name.
      \item The variables
        \renderPartialCode{258}{259}
        are probably swapped and the name result so misleading. In fact, the value of quantities is stocked into
        the variable \textbf{occs} and vice-versa with the Occurrences.
      \item The method as the class name
        \renderPartialCode{288}{288}
        \renderPartialCode{299}{299}
        should be renamed respectively in \textbf{productOrderInMap} and \textbf{ProductInMapComparator} because nothing
        in the code suggest the the ordering method is based on a map but it is instead apply on a map structure.
    \end{itemize}
  \itemBold{Checklist[5]}
    \begin{itemize}
      \item The variable' name, even if it is not a method, should be written with some separator or upper cases for
        a more clear writing.
        \renderPartialCode{72}{72}
    \end{itemize}
  \itemBold{Checklist[6]}
    \begin{itemize}
      \item The variables' name
        \renderPartialCode{300}{301}
        should contain an under-score because it is a class' attribute.
    \end{itemize}
  \end{itemize}
\end{itemize}
\section{Indentation}
  \begin{itemize}
    \itemBold{Checklist[8]}
      \begin{itemize}
        \item The series of if are not only without parenthesis but are also not correctly indented
          \renderPartialCode{66}{66}
          \renderPartialCode{154}{155}
          \renderPartialCode{189}{189}
          \renderPartialCode{192}{192}
          \renderPartialCode{197}{197}
          \renderPartialCode{210}{210}
          \renderPartialCode{289}{290}
          \renderPartialCode{323}{323}
      \end{itemize}
  \end{itemize}
\section{Braces}
\begin{itemize}
  \itemBold{Checklist[10]}
    \begin{itemize}
      \item A consistence braces is used, in particular the \textbf{Kernighan and Ritchie} style is used.
    \end{itemize}
  \itemBold{Checklist[11]}
    \begin{itemize}
      \item The series of if not correctly branched
        \renderPartialCode{66}{66}
        \renderPartialCode{154}{155}
        \renderPartialCode{189}{189}
        \renderPartialCode{192}{192}
        \renderPartialCode{197}{197}
        \renderPartialCode{210}{210}
        \renderPartialCode{289}{290}
        \renderPartialCode{323}{323}
    \end{itemize}
\end{itemize}
\section{File Organization}
\begin{itemize}
  \itemBold{Checklist[12]}
    \begin{itemize}
      \item There are blank line that are not expected or even worthless as:
        \begin{itemize}
          \itemBold{@71}
          \itemBold{@78}
          \itemBold{@184}
          \itemBold{@188}
          \itemBold{@196}
          \itemBold{@269}
          \itemBold{@331}
        \end{itemize}
        The partial code is not showed but the reader can use the class furnished bellow as reference, one example is
        reported above.
        \renderPartialCode{70}{74}
    \end{itemize}
    \itemBold{Checklist[13]-Checklist[14]}
      \begin{itemize}
        \item A lot of line passed the 80 characters, here are reported only the line that pass the 120 character
        and that make the reading difficult
          \begin{itemize}
            \itemBold{@76}
            \itemBold{@77}
            \itemBold{@79}
            \itemBold{@84}
            \itemBold{@121}
            \itemBold{@158}
            \itemBold{@159}
            \itemBold{@160}
            \itemBold{@169}
            \itemBold{@214}
            \itemBold{@215}
            \itemBold{@216}
            \itemBold{@260}
            \itemBold{@265}
            \itemBold{@268}
            \itemBold{@288}
          \end{itemize}
          The partial code is not showed but the reader can use the class furnished bellow as reference, one example is
          reported above.
          \renderPartialCode{70}{74}
      \end{itemize}
\end{itemize}
\section{Comments}
\begin{itemize}
  \itemBold{Checklist[18]}
    \begin{itemize}
      \item The comment
        \renderPartialCode{56}{68}
      is supposed to precede documentation or method comment but in fact it does nothing.
      \item The comment
        \renderPartialCode{76}{76}
        is not clear and does not explain anything about the code above.
      \item The comment
        \renderPartialCode{174}{174}
        has to be placed before the start of the while to be more helpful.
    \end{itemize}
\end{itemize}
\section{Java Source File}
\begin{itemize}
  \itemBold{Cheklist[23]}
    \begin{itemize}
      \item The \textbf{Javadoc} is implemented for the assigned class but it is not really helpful because
        it contains only the definitio of the method without additional comment to the functionalities of the class itself.
    \end{itemize}
\end{itemize}
\section{Class and Interface Declarations}
\begin{itemize}
  \itemBold{Checklist[25a]}
    \begin{itemize}
      \item There are not comments about the class or the class' methods.
    \end{itemize}
  \itemBold{Checklist[25b]}
    \begin{itemize}
      \item The interface' method
      \renderPartialCode{327}{337}
      must be written before the class' variables and the class' methods
    \end{itemize}
  \itemBold{Checklist[26]}
    \begin{itemize}
      \item The methods are not grouped in any class or sub-class.
    \end{itemize}
  \itemBold{Checklist[27]}
    \begin{itemize}
      \item The first method of the class
        \renderPartialCode{61}{63}
        is particularly long. Especially because it could be reduce in different sub-methods in order to make the class
        easier to debug and maintain.
    \end{itemize}
\end{itemize}
\section{Initialization and Declarations}
\begin{itemize}
  \itemBold{Checklist[28]}
    \begin{itemize}
      \item The methods
        \renderPartialCode{288}{288}
        \renderPartialCode{328}{328}
        cannot be public as the class is private
    \end{itemize}
  \itemBold{Checklist[33]}
    In this section we speak about declaration, not the initialisation.
    \begin{itemize}
      \item The declaration
        \renderPartialCode{68}{68}
        has to be moved before the first if.
      \item The declarations
        \renderPartialCode{174}{175}
        have to be moved before the if, according to the checklist.
      \item The declarations in this code
        \renderPartialCode{187}{196}
        have to be move together at the start of the block even if the reading is easier.
      \item The declarations
        \renderPartialCode{225}{225}
        \renderPartialCode{236}{237}
        \renderPartialCode{249}{249}
        \renderPartialCode{253}{255}
        \renderPartialCode{268}{268}
        \renderPartialCode{292}{292}
        have to be moved at the start of the method;
    \end{itemize}
\end{itemize}
\section{Object Comparison}
\begin{itemize}
  \itemBold{Checklist[40]}
    \begin{itemize}
      \item The comparison between element must be done with the method \textbf{equal()}
        \renderPartialCode{66}{66}
        \renderPartialCode{127}{127}
        \renderPartialCode{154}{155}
        \renderPartialCode{162}{162}
        \renderPartialCode{189}{189}
        \renderPartialCode{192}{192}
        \renderPartialCode{197}{197}
        \renderPartialCode{289}{290}
        \renderPartialCode{323}{323}
        \renderPartialCode{332}{332}
    \end{itemize}
\end{itemize}
\section{Exceptions}
\begin{itemize}
  \itemBold{Checklist[52]}
    \begin{itemize}
      \item Also the interaction with the object \textbf{Request} should be put into the try catch
        \renderPartialCode{62}{64}
        \renderPartialCode{149}{152}
    \end{itemize}
  \itemBold{Checklist[53]}
    \begin{itemize}
      \item The exception are threaded only generally so it is not possible to evaluate this point without a proprerly
      documentation.
    \end{itemize}
\end{itemize}
