
The project is subject to a series of risks with different levels of danger.
In order to prevent so in necessary to have an \textbf{Proactive Risk Management Strategy} that could prevent or eventually solve some
of the major risk. In order to build a good \textbf{Proactive Risk Management Strategy} we describe in this chapter the potential risk
the project is subject to, categorysing them and proposing a solution for each main category. Business risk are not take into account
in this chapter as we presume an carefull analysis of the market, budget and business strategy has already be done taking in account also
the related risk.
\begin{itemize}
    \itemBold{Project Risk}
      \begin{itemize}
        \item The work force available is less than expected. It could be caused by developers’ lack of experience in programming and in project planning and management.
        it could potentially slow down the overall project development thus leading to the risk of an extension of some deadline. A solution for this problem could be,
        in scheduling phase, take into account that possibility and assign a bit more effort than needed to each task.
      \end{itemize}
    \itemBold{Technical Risk}
      \begin{itemize}
        \item Faults in reusable software components have to be repaired before these components are reused. It could be prevented by a
        develop strong unit tests, in fact, it reduces the error' probability and in the worst case the reparation time is minimized.
        \item The database and other external software components used in the system cannot process as many operations per second as expected. It could be due to
        a under-estimation of the city dimension or a bigger amount of users than expected. A possible solution could be over-estimate the operations throughput in order to choose the appropriate support infrastructure.
      \end{itemize}
\end{itemize}
