The Functional Point approach is a technique that allows to evaluate the effort
needed for the design and implementation of a project. We have used this technique
to evaluate the application dimension basing on the functionalities of the
application itself. The functionalities list has been obtained from the RASD document
and for each one of them the realization complexity has been evaluated.
The functionalities has been groped in:
\begin{itemize}
\item Internal Logic File: it represents a set of homogeneous data handled by the system
\item External Interface File: it represents a set of homogeneous data used
by the application but handled by external application
\item External Input: elementary operation that allows input of data in the system
\item External Output: elementary operation that creates a bit stream towards
the outside of the application
\item External Inquiry: elementary operation that involves input and output operations
\end{itemize}

\newpage

The following table outline the number of Functional Point based on functionality
and relative complexity:

\begin{table}[h]
	\centering
	\begin{tabular}{| l | l | l | l |}
		\hline
		\multirow{2}{*}{\textbf{Function Type}} & \multicolumn{3}{c|}{\textbf{Complexity}} \\
		\cline{2-4}
		& Simple & Medium & Complex  \\
		\hline
		Internal Logic File & 7 & 10 & 15 \\ \hline
		External Interface File & 5 & 7 & 10 \\ \hline
		External Input & 3 & 4 & 6 \\ \hline
		External Output & 4 & 5 & 7 \\ \hline
		External Inquiry & 3 & 4 & 6 \\ \hline

	\end{tabular}
\end{table}


\section{Internal Logic File}
Internal Logic Files are homogeneous sets of data used and managed by the application.\\
Summary of the calculated weights:
\begin{equation*}
	\begin{aligned}
		&	FP_{ILF}
		& & = 5 \cdot w_{Simple,ILF} + 3 \cdot w_{Medium,ILF} + 1 \cdot w_{Complex,ILF}\\
		&&& = 5 \cdot 7 + 3 \cdot 10 + 1 \cdot 15\\
		&&& = 80\\
	\end{aligned}
\end{equation*}
In details:
\begin{itemize}
	\fp{Ride}{Ride data. Frequent modifications, frequent insertions, frequent readings.}{1}{w_{Medium,ILF}}
  \fp{User}{Personal Registered Passenger data. Infrequent modifications, but frequent insertions and readings.}{1}{w_{Medium,ILF}}
  \fp{Car}{Car information. Infrequent insertion and modification, frequent reading}{1}{w_{Medium,ILF}}
	\fp{Customer Service}{Customer help request. Very infrequent modifications, insertions and readings.}{1}{w_{Simple,ILF}}
	\fp{Payment}{Data of the ended rent. Infrequent modification and reading, very frequent insertion}{1}{w_{Complex,ILF}}
	\fp{Payment Method}{User's payment data, Frequent readings, infrequent modification and insertion}{1}{w_{Simple,ILF}}
	\fp{Parking}{Parking data, frequent reading, infrequent insertion and modifications}{1}{w_{Simple,ILF}}
	\fp{Service Station}{Service station data. frequent reading, infrequent insertion and modifications}{1}{w_{Simple,ILF}}
	\fp{GPS Data}{GPS Data, Database Server side. Very frequent insertions and readings, but very infrequent modifications.}{1}{w_{Simple,ILF}}
\end{itemize}
%
\section{External Interface File}
External Interface Files are homogeneous sets of data used by the application but generated and maintained by other applications.\\
Summary of the calculated weights:
\begin{equation*}
	\begin{aligned}
		&	FP_{ELF}
		& & = 2 \cdot w_{Simple,ELF}\\
		&&& = 2 \cdot 5\\
		&&& = 10\\
	\end{aligned}
\end{equation*}
\newpage{}
In details:
\begin{itemize}
\fp{Google Maps API}{Google Maps API related data about Travel Time and Addresses. Very frequent readings.}{1}{w_{Simple,ELF}}
\fp{Paypal API}{Paypal API request related to user's payement. Frequent insertion}{1}{w_{Simple,ELF}}
\end{itemize}
%
\section{External Input}
External Inputs are elementary operations to elaborate data coming from the external environment.\\
Summary of the calculated weights:
\begin{equation*}
	\begin{aligned}
		&	FP_{EI}
		& & = 5 \cdot w_{Simple,EI} + 2 \cdot w_{Medium,EI} + 2 \cdot w_{Complex,EI}\\
		&&& = 5 \cdot 3 + 2 \cdot 4 + 2 \cdot 6\\
		&&& = 35\\
	\end{aligned}
\end{equation*}
In details:
\begin{itemize}
	\fp{Login}{This operation requires a simple effort. In fact it has to perform few steps in order to conclude the procedure.}{1}{w_{Simple,EI}}
	\fp{Logout}{This operation requires a simple effort. In fact it has to perform few steps in order to conclude the procedure.}{1}{w_{Simple,EI}}
	\fp{Registration}{This operation requires a simple effort. In fact it has to perform few steps in order to conclude the procedure.}{1}{w_{Simple,EI}}
	\fp{Handle Personal Profile}{This operation requires a simple effort. In fact it has to perform few steps in order to conclude the procedure.}{1}{w_{Simple,EI}}
	\fp{Car Interaction}{This operation requires a medium effort. In fact it is very frequent.}{1}{w_{Medium,EI}}
	\fp{Car Reservation}{This operation requires a complex effort. In fact it has to perform many elementary steps in order to handle it}{1}{w_{Complex,EI}}
	\fp{Problem}{This operation requires a medium effort. In fact it has to perform many elementary steps in order to handle the problem.}{1}{w_{Medium,EI}}
  \fp{Rent}{This operation requires a simple effort.}{1}{w_{Simple,EI}}
  \fp{End of Rent}{This operation requires a complex effort. In fact it has to perform many elementary steps in order to handle it}{1}{w_{Complex,EI}}
\end{itemize}
%
\section{External Output}
External Outputs are elementary operations that generate data for the external environment, and they usually include the elaboration of data from logic files.\\
Summary of the calculated weights:
	\begin{equation*}
		\begin{aligned}
		&	FP_{EO}
		& & = 4 \cdot w_{Simple,EO}\\
		&&& = 4 \cdot 4\\
		&&& = 16\\
	\end{aligned}
\end{equation*}
In details:
\begin{itemize}
	\fp{User Notification}{The result of the end of the rent must be send to the user who handle it}{1}{w_{Simple,EO}}
	\fp{Email Confirmation}{The result of this operation must be sent to the user that registers}{1}{w_{Simple,EO}}
	\fp{Maintenance Notification}{The result of the maintenance call must be sent to the specific user.}{1}{w_{Simple,EO}}
	\fp{Reservation}{Informations about the reservation must be sent to the related user}{1}{w_{Simple,EO}}
\end{itemize}
%
\section{External Inquiry}
External Inquiries are elementary operations that involve input and output, without significant elaboration of data from logic files.\\
Summary of the calculated weights:
\begin{equation*}
	\begin{aligned}
		&	FP_{EIQ}
		& & = 2 \cdot w_{Simple,EIQ}\\
		&&& = 2 \cdot 3\\
		&&& = 6\\
	\end{aligned}
\end{equation*}
In details:
\begin{itemize}
	\fp{Manager car status}{In order to perform these operations the system has only to retrieve, send and render simple data.}{1}{w_{Simple,EIQ}}
	\fp{Manager user status}{In order to perform these operations the system has only to retrieve, send and render simple data.}{1}{w_{Simple,EIQ}}
\end{itemize}
%
\section{Summary}
All the calculated $FP_{i}$ sums up to $FP$, which is the total Function Points value:
\begin{equation*}
	\begin{aligned}
		&	FP
		& & = FP_{ILF} + FP_{ELF} + FP_{EI} + FP_{EO} + FP_{EIQ}\\
		&&& = 80 + 10 + 35 + 16 + 6\\
		&&& = 147\\
	\end{aligned}
\end{equation*}
The total $FP$ value is then multiplied by a constant factor $k_{i,j}$ that depends on the programming language $i$ used to develop the software and the company gearing ratio $j$.\par
The gearing ratio is the level of a company's debt related to its equity capital, usually expressed in percentage form.
Gearing is a measure of a company's financial leverage and shows the extent to which its operations are funded by lenders versus shareholders.\par
This final calculation gives us the number of SLOC $n_{SLOC}$ estimated for \PowerEnJoy{}:
\begin{equation*}
	\begin{aligned}
		&   n_{SLOC}
		& & = FP \cdot k_{Java, Avg}\\
		&&& = 147 \cdot 53\\
		&&& = 7791 \text{ SLOC}
	\end{aligned}
\end{equation*}
%
